Given an undirected graph $G = (V, E)$ and a weight function $w:E \to \R$, 
the \Problem{} problem asks to find a minimum weight sub-tree of $G$, 
$T = (U, F)$, such that every $v \in V \setminus U$ is adjacent to at least one 
vertex in $U$.
The special case when the weight function is uniform is known as the 
\textsc{Minimum Connected Dominating Set} problem.

Given an undirected graph $G = (V, E)$ with some subsets of vertices called groups,
and a weight function $w:E \to \R$,
the \textsc{Group Steiner Tree} problem is to find a minimum weight sub-tree
of $G$ which contains at least one vertex from each group. 

In this paper we show that the two problems are equivalents 
from approximability perspective.
This improves upon both the best known approximation algorithm and the best 
inapproximability result for the \Problem{} problem.
We also consider two extrema variants of the \Problem{} problem, namely,
the \ProblemStar{} and the \ProblemPath{} problems 
which ask to find a minimum dominating star and path respectively.  
