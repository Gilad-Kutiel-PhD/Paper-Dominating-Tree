We start by showing an approximation-preserving reduction from \Prob{} to
\ProbGroup{}.
Given an instance of \Prob{}, $(G, w)$, where $G = (V, E)$, 
we define an instance of \ProbGroup{},
$(G, w, V, \mathcal{G})$.
We now have to define $\mathcal{G}$: for each vertex $v \in V$ we define 
$g_v \in \mathcal{G}$ to be $\{v\} \cup N(v)$.
Figure~\ref{fig:prob-leq-group} depicts this transformation.
Note that the number of groups is $n$ and the that the size of the largest group is $\Delta$.
Clearly this transformation can be done in polynomial time.
The following two claims show that this is an approximation preserving reduction. 

\begin{figure}
\begin{center}
\begin{tikzpicture}[every node/.style={default node}]
\foreach[count=\i] \x \y in {
	0/0
	,1/2,1/-1
	,2/0
	,3/1,3/-2
	,4/0
}{
	\node(\i) at(\x,\y) {\i};
}

\foreach \u \v in {
	1/2,1/3%
	,2/4,2/5%
	,3/4,3/6,3/7%
	,4/5,4/7%
	,5/7%
	,6/7%
}{
	\draw (\u) -- (\v);
}

\begin{scope}[xshift=6cm]
\foreach[count=\i] \x \y in {
	0/0
	,1/2,1/-1
	,2/0
	,3/1,3/-2
	,4/0
}{
	\node(\i) at(\x,\y) {\i};
}

\foreach \u \v in {
	1/2,1/3%
	,2/4,2/5%
	,3/4,3/6,3/7%
	,4/5,4/7%
	,5/7%
	,6/7%
}{
	\draw (\u) -- (\v);
}
\end{scope}

\draw[dashed, blue]
(2.north) to[out=0,in=90]
(5.east) to[out=-90,in=0]
(4.south) to[out=180,in=-90]
(1.west) to[out=90,in=180]
(2.north)
;

\end{tikzpicture}
\end{center}
\caption{\label{fig:prob-leq-group}
From left to right:
a) A \Prob{} instance, weights are omitted.
b) A corresponding \ProbGroup{} instance on the same weighted graph.
For each vertex $v$ we define a group
that contains its neighborhood to ensure that in any group Steiner tree there is at least
one vertex that dominate $v$.
$g_2$ is marked in the figure with a dashed blue line.  
}
\end{figure}

\begin{claim}
Any dominating tree, $T$, in $(G, w)$ is a feasible group Steiner tree in $(G, w, V, \mathcal{G})$.
\end{claim}

\begin{proof}
Assume for contradiction that $T$ is not feasible group Steiner tree, that is, there is 
a group $g_v$ such that none of the vertices in $g_v$ is spanned by $T$, that is $v$
is not in $T$ nor any of its neighbors and thus $T$ is not a dominating tree - contradiction. 
\end{proof}
 
\begin{claim}
Any feasible group Steiner tree in $(G, w, V, \mathcal{G})$, $T$, is a dominating tree
in $(G, w)$.
\end{claim}

\begin{proof}
Assume for contradiction that $T$ is not a dominating tree, that is, there is 
a vertex $v$ not in $T$ such that none of its neighbors belong to $T$, thus, 
$T$ does not cover $g_v$ - contradiction. 
\end{proof}
