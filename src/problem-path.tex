We show that the \ProblemPath{} problem (\ProbPath{}) cannot be approximated at all
unless $P = NP$.
We show a reduction from the \ProblemHam{} problem (\ProbHam{}).
In \ProbHam{} we are given an undirected graph $G = (V, E)$ and we are 
asked to decide if there is an Hamiltonian path 
(a simple path traversing all the vertices in $V$) in $G$ or not.
\ProbHam{} is one of the classical NP-hard problems.

Given an instance of \ProbHam{}, $G = (V, E)$, we define an instance of \ProbPath{},
$(G', w)$, where $G' = (V \cup \{v' : v \in V\}, E \cup \{vv' : v \in V\})$.
We set $w(e) = 0$ for every edge $e \in E$ and set $w(e) = \infty$ otherwise.
Figure~\ref{fig:hamiltonian} depicts this transformation.  
We now claim that any (multiplicative) approximation algorithm for \ProbPath{}
can solve \ProbHam{}.
Let $G$ be an instance of (decision problem) \ProbHam{}, 
and denote by $A(G', w)$ the value of (approximation) algorithm, $A$, 
on the corresponded \ProbPath{} instance, then:

\begin{claim}
$A(G', w) = 0 \iff G \in \ProbHam{}$. 
\end{claim} 

\begin{proof}
Let $P$ be a dominating path in $(G', w)$ with value 0, 
then it uses only edges of $E$, moreover $P$ is an hamiltonian path in $G$ or otherwise
there is a vertex $v'$ that is not dominated by $P$.
Now, let $P$ be an hamiltonian path in $G$ then $P$ is a dominating path in $G'$ with value
0, thus, any (multiplicative) approximation algorithm must also find a path value 0.
\end{proof}


\begin{figure}
\begin{center}
\begin{tikzpicture}[every node/.style={default node}]
\foreach[count=\i] \x \y in {
	0/0
	,1/2,1/-1
	,2/0
	,3/1,3/-2
	,4/0
}{
	\node(\i) at(\x,\y) {\i};
}

\foreach \u \v in {
	1/2,1/3%
	,2/4,2/5%
	,3/4,3/6,3/7%
	,4/5,4/7%
	,5/7%
	,6/7%
}{
	\draw (\u) -- (\v);
}

\begin{scope}[xshift=6cm]
\foreach[count=\i] \x \y in {
	0/0
	,1/2,1/-1
	,2/0
	,3/1,3/-2
	,4/0
}{
	\node(\i) at(\x,\y) {\i};
}

\foreach \u \v in {
	1/2,1/3%
	,2/4,2/5%
	,3/4,3/6,3/7%
	,4/5,4/7%
	,5/7%
	,6/7%
}{
	\draw (\u) -- (\v);
}

\foreach[count=\i] \p in {
	left,above,below left,left,above right,below,right%
}{
	\node(\i')[\p=3mm of \i] {\i'};
	\draw[dashed] (\i) -- (\i');
}
\end{scope}
\end{tikzpicture}
\end{center}
\caption{\label{fig:hamiltonian}
From left to right
a) An instance of the \ProblemHam{} problem.
b) The corresponded \ProblemPath{} instance, 
original edges have zero weight, dashed edges have infinite weight.
}
\end{figure} 