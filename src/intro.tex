Given an undirected graph $G = (V, E)$ and a weight function $w:E \to \R$, 
the \Problem{} problem (\Prob{}) asks to find a minimum weight sub-tree of $G$, 
$T = (U, F)$, such that every $v \in V \setminus U$ is adjacent to at least one 
vertex in $U$.
The special case when the weight function is uniform is known as the 
\textsc{Minimum Connected Dominating Set} problem (CDS). 
Both CDS and \Prob{} have many applications in routing
for mobile ad-hoc networks, see for 
example~\cite{shin2010approximation,cheng2003polynomial,das1997routing,adasme2016models,adasme2017minimum}.
Figure~\ref{fig:problem} shows an example instance and a possible solution to the problem. 

\begin{figure}
\begin{center}
\begin{tikzpicture}[every node/.style={default node}]
\foreach[count=\i] \x \y in {
	0/0
	,1/2,1/-1
	,2/0
	,3/1,3/-2
	,4/-1
}{
	\node(\i) at(\x,\y) {\i};
}

\foreach \u \v \w in {
	1/2,1/3%
	,2/4,2/5%
	,3/4,3/6,3/7%
	,4/5,4/7%
	,5/7%
	,6/7%
}{
	\draw (\u) -- (\v) node[label above]{\w};
}

\begin{scope}[xshift=6cm]
\foreach[count=\i] \x \y in {
	0/0
	,1/2,1/-1
	,2/0
	,3/1,3/-2
	,4/-1
}{
	\node(\i) at(\x,\y) {\i};
}

\foreach \u \v \w in {
	1/2,1/3%
	,2/4,2/5%
	,3/4,3/6,3/7%
	,4/5,4/7%
	,5/7%
	,6/7%
}{
	\draw (\u) -- (\v) node[label above]{\w};
}
\end{scope}
\end{tikzpicture}
\end{center}
\caption{\label{fig:problem}
From left to right
a) An instance of the \Problem{} problem.
b) A possible solution (solid edges) with value of 17.
}
\end{figure}

Given an undirected graph $G = (V, E)$ with some subsets of vertices called groups,
and a weight function $w:E \to \R$,
the \textsc{Group Steiner Tree} problem is to find a minimum weight sub-tree
of $G$ which contains at least one vertex from each group. 

In this paper we show that the two problems are equivalents 
from approximation algorithms perspective.
This improves upon both the best known approximation algorithm and the best 
inapproximability result for \Prob{}.
We also consider two extrema variants of \Prob{}, namely,
the \ProblemStar{} (\ProbStar{}) and the \ProblemPath{} (\ProbPath{}) 
problems which ask to find a minimum dominating star and path respectively.  

\textbf{Previous Work:}
CDS has a long history starting at the late 70s~\cite{sampathkumar1979connected},
and it is approximable within $\ln\Delta + 3$~\cite{guha1998approximation} 
where $\Delta$ is the maximum degree in $G$,
which is the best one can wish for if $P \neq NP$. 

\Prob{}, to the best of our knowledge, was introduced in~\cite{shin2010approximation}.
In the same paper it was shown that the \textsc{Minimum Weighted Dominating Set} problem 
can be reduced to \Prob{} in a way that preserve the approximation ratio.
Thus, there is no $c\log n$-approximation algorithm, for some $c > 0$, for \Prob{}
unless $P = NP$.
In the same paper it was shown that \Prob{} can be reduced to the 
\textsc{Minimum Directed Steiner Tree} problem in a way that preserve the approximation ratio.
Unfortunately, the current best approximation algorithm for the 
\textsc{Minimum Directed Steiner Tree} problem yields a $|S|^\varepsilon$ 
approximation ratio~\cite{charikar1999approximation},
where $S$ is the set of terminals.
To the best of our knowledge, this is the best approximation algorithm known for \Prob{}.
The existence of a dominating path in a graph was studied in several 
papers see~\cite{broersma1988existence, faudree2017minimum} for example,
but, to the best of our knowledge, it was never considered from algorithmic perspective.

\textbf{Our Result:}
We show that the \Problem{} problem is equivalent to the \ProblemGroup{} problem
from approximability perspective, 
by doing so we prove that \Prob{} is inapproximable within $\log^{2 - \varepsilon} n$   
unless $NP \subseteq \text{ZTIME}(n^{\text{polylog}(n)})$.
This also, directly leads to a $\log^2 n \log \Delta$-approximation algorithm, 
where $\Delta$ is the maximum degree in $G$.
We also consider the \ProblemStar{} problem, show that it is inapproximable within
ratio of $c \log n$ for some $c > 0$ and show how to reduce it to the \ProblemSetCover{}
problem to obtain a $O(\log n)$ approximation.
Finally, we consider the \ProblemPath{} problem and show that it is inapproximable at all.