% \begin{algorithm}[H]
% 	\KwData{this text}
% 	\KwResult{how to write algorithm with \LaTeX2e }
% 	initialization\;
% 	\While{not at end of this document}{
% 		read current\;
% 		\eIf{understand}{
% 			go to next section\;
% 			current section becomes this one\;
% 		}{
% 			go back to the beginning of current section\;
% 		}
% 	}
% \caption{How to write algorithms}
% \end{algorithm}
% 

Given an undirected graph $G = (V, E)$ and a weight function $w:E \to \R$, 
the \Problem{} problem (\Prob{}) asks to find a minimum weight sub-tree of $G$, 
$T = (U, F)$, such that every $v \in V \setminus U$ is adjacent to at least one 
vertex in $U$.
The special case when the weight function is uniform is known as the 
\textsc{Minimum Connected Dominating Set} problem (CDS). 
Both CDS and \Prob{} have many applications in routing
for mobile ad-hoc networks, see for example~\cite{
shin2010approximation%
,cheng2003polynomial%
,das1997routing%
,adasme2016models%
,adasme2017minimum%
}.


\textbf{Previous Work:}
CDS has a long history starting at the late 70s~\cite{sampathkumar1979connected},
and it is approximable within $\ln\Delta + 3$~\cite{guha1998approximation} 
where $\Delta$ is the maximum degree in $G$.
We are not aware of any hardness result regarding the CDS that show that this approximation
ratio is the best we can hope for.

\Prob{}, to the best of our knowledge, was introduced in~\cite{shin2010approximation}.
In the same paper it was shown that the \textsc{Minimum Weighted Dominating Set} problem 
can be reduced to \Prob{} in a way that preserve the approximation ratio.
Thus, there is no $(1 - \varepsilon)\log n$-approximation algorithm for the \Prob{} problem
unless $P = NP$.
In the same paper it was shown that \Prob{} can be reduced to the 
\textsc{Minimum Directed Steiner Tree} problem in a way that preserve the approximation ratio.
Unfortunately, the current best approximation algorithm for the 
\textsc{Minimum Directed Steiner Tree} problem yields a $|S|^\varepsilon$ 
approximation ratio~\cite{charikar1999approximation},
where $S$ is the set of terminals.
It is also known that this problem cannot be approximated within ratio 
$\log^{2 - \varepsilon}$ 
unless $NP \subseteq \text{ZTIME}(n^{\text{polylog}(n)})$~\cite{halperin2003polylogarithmic}.
To the best of our knowledge, this is the best approximation algorithm known for \Prob{}.
     
